In their paper \cite{bib11}, Li et al describe an interpolation technique to generate texture from two input style images. According to their method, which we shall refer to as \textit{Original-Merge}, at each pipeline level two WCT calls are conducted: $f_{cs_1} = WCT(x, f_{s_1}), f_{cs_2} = WCT(x, f_{s_2})$ where $x$ is the content features output by the level encoder. The results are then interpolated with $f_{cs} = \beta f_{cs_1} + (1-\beta)f_{cs_2}$. After the interpolation, $f_{cs}$ is reconstructed as usual using the level decoder.\\

In this section we researched other merging techniques to merge two style with a \textbf{single} WCT call for each pipeline level. We present the following three techniques:

\begin{itemize}
	\item \textbf{Level Merge}: ...
	\item \textbf{Channel Merge}: ...
	\item \textbf{Interpolate-Style Merge}: ...
	
\end{itemize}
%show an improved algorithm which merges two style images bu using WCT algorithm which based on singular value decomposition (SVD). Here, we both implement the original merge algorithm as proposed in \cite{bib11} as well as introduce three additional efficient methods based on the use WCT.