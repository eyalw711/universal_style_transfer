The UST algorithm as described in \cite{bib11} formulates style transfer as an image reconstruction process coupled with feature transformation of whitening and coloring (WCT). The reconstruction step is responsible for inverting features back to the RGB space, the feature transformation step matches the feature statistics of a content image to a style image. The file \texttt{universal\_style\_transfer.py} implements the UST algorithm, use '\texttt{python universal\_style\_transfer.py -h}' for help.

\subsubsection{Architecture}
The basic building block of the UST is a pair of VGG based encoder and decoder, explained and described in section ~\ref{models_methods_lbl}. A single level stylization pass in UST would be to pass a content image and a style image through an encoder, perform a transformation called WCT on the extracted content features, then reconstruct the output of the WCT using the decoder. The image output by the decoder should have both the original content and some notion of the artistic styles in the style image.\\ To achieve better results, UST does more than a single level stylization pass, it constructs a pipeline of such levels, each with its own encoder-decoder pair. Passing the content through this pipeline transfers style in different feature depths, so the result is generally more pleasing. See figure ~\ref{fig:full-pipeline} for a block diagram of the pipeline pass to perform UST on a pair of images $c,s$. In the following sections, for the $j$-th level of the pipeline denote the encoder by $E_j$ and the decoder by $D_j$.

\begin{figure}[h!]
	\centering
	\includegraphics[width=0.5\linewidth]{UST_arc_mlt_level_pipeline.png}
	\caption{Universal Style Transfer pipeline architecture. Each level of stylization consists of single encoder-WCT-decoder network with different decreasing number of VGG layers. C and S are content and style images, respectively.
	}
	\label{fig:full-pipeline}
\end{figure}

\subsubsection{WCT} As explained above, given a pair of content image $I_C$ and style image $I_S$ as input for the $j$-th level, at first the encoder extracts their features by $f_c = E_j(I_C)$, $f_s = E_j(I_S)$. Notice that both these features are three dimensional, $f_c$ is $[c\times h \times w]$ and $f_s$ is $[c\times h' \times w']$. Their height and width are generally different but their number of channels, $c$, is the same.\\
The WCT is a pair of projection functions working on vectorized VGG features. The term \textit{vectorized} means that we treat every channel of the feature tensor as a row vector, so $\vec{f_c}$ is $[c \times hw]$ and $\vec{f_s}$ is $[c \times h'w']$. WCT works on the features extracted by the encoder, and its result is fed to the decoder, as seen in Figure ~\ref{fig:full-pipeline}. WCT is a composition of a whitening transform ($P_C$) and a coloring transform ($P_S$). The key idea behind the WCT is to directly match feature correlations of the content image to those of the style image via those two projections. Specifically, the WCT transform is applied to the vectorized content feature $\vec{f_c}$ via:
\begin{equation}
f_{cs} = P_S P_C \vec{f_c}
\end{equation}
Where $P_C=E_C\Lambda_C^{-\frac{1}{2}}$, and $P_S=E_S\Lambda_S^{\frac{1}{2}}$. Here $\Lambda_C$ and $\Lambda_S$ are the diagonal matrices with the eigenvalues of the covariance matrix $\vec{f_c} \vec{f_c}^T [c\times c]$ and $\vec{f_s} \vec{f_s}^T [c\times c]$ respectively. The matrices $E_C$ and $E_S$ are the corresponding orthonormal matrices of eigenvectors, respectively, (see figure ~\ref{fig:WCT-vis}). After the transformation, the correlations of transformed content features match those of the style, i.e., $\vec{f_{cs}} \vec{f_{cs}}^T = \vec{f_s} \vec{f_s}^T$.
The WCT performs well for artistic image stylization. However it generates
structural artifacts (e.g., distortions on object boundaries)
WCT is implemented in the function \texttt{wct} in file \texttt{utils.py}, \textcolor{red}{and explained in greater detain in section 3.2 in \cite{bib11}}.

\begin{figure}[h!]
	\centering
	\begin{subfigure}[b]{0.4\linewidth}
		\includegraphics[width=\linewidth]{whitening.png}
		\caption{Data whitening}
	\end{subfigure}
	\begin{subfigure}[b]{0.4\linewidth}
		\includegraphics[width=\linewidth]{coloring.png}
		\caption{Data coloring}
	\end{subfigure}
	\caption{Whitening-Coloring-Transformation}
	\label{fig:WCT-vis}
\end{figure}