In this project we implemented the Universal Style Transfer algorithm as a lightweight PyTorch CLI tool which is ready to be used by users. Additionally, since this tool supports hot replacement of its encoder-decoder models, researchers may use our tool to test the UST algorithm with different sets of models. Furthermore, our tool contains upgrades to the original algorithm in two aspects: Boost and Merge. The Boost step is an innovative procedure to enhance the style transfer effect on the result image, it has a practical application and produces better results. In the merging aspect, we offered three innovative techniques to perform style transfer with two style images, all of which are more time-efficient than the original technique in UST. In future research, it is possible to study extensions of our merging techniques that would allow for UST with more than a pair of style images. Additionally it is worthwhile to study the effect of multiple boost steps where we conducted one.

%training - more resources - or smaller models\\
%we implemented UST as a lightweight tool ready to use by users + available interface for switching models by replacing the ref\_models\_factory.py \\
%innovative new step for the UST algorithm: Boost, that can have practical applications. Learned that adding channels and downsizing h,w increases the style transfer effect without harming the content\\
%innovative new merging techniques which save runtime and feature results on par with the original technique. - Learned that the SVD is a major time consumer in the UST and reducing calls to SVD may reduce up to 7\% in the program runtime during merge.\\

%suggest ideas:\\
%formulate merging techniques for merging more than two styles.
%research the effect of multiple boost effects per pipeline level.
